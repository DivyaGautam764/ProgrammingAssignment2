% Options for packages loaded elsewhere
\PassOptionsToPackage{unicode}{hyperref}
\PassOptionsToPackage{hyphens}{url}
%
\documentclass[
]{article}
\usepackage{lmodern}
\usepackage{amssymb,amsmath}
\usepackage{ifxetex,ifluatex}
\ifnum 0\ifxetex 1\fi\ifluatex 1\fi=0 % if pdftex
  \usepackage[T1]{fontenc}
  \usepackage[utf8]{inputenc}
  \usepackage{textcomp} % provide euro and other symbols
\else % if luatex or xetex
  \usepackage{unicode-math}
  \defaultfontfeatures{Scale=MatchLowercase}
  \defaultfontfeatures[\rmfamily]{Ligatures=TeX,Scale=1}
\fi
% Use upquote if available, for straight quotes in verbatim environments
\IfFileExists{upquote.sty}{\usepackage{upquote}}{}
\IfFileExists{microtype.sty}{% use microtype if available
  \usepackage[]{microtype}
  \UseMicrotypeSet[protrusion]{basicmath} % disable protrusion for tt fonts
}{}
\makeatletter
\@ifundefined{KOMAClassName}{% if non-KOMA class
  \IfFileExists{parskip.sty}{%
    \usepackage{parskip}
  }{% else
    \setlength{\parindent}{0pt}
    \setlength{\parskip}{6pt plus 2pt minus 1pt}}
}{% if KOMA class
  \KOMAoptions{parskip=half}}
\makeatother
\usepackage{xcolor}
\IfFileExists{xurl.sty}{\usepackage{xurl}}{} % add URL line breaks if available
\IfFileExists{bookmark.sty}{\usepackage{bookmark}}{\usepackage{hyperref}}
\hypersetup{
  hidelinks,
  pdfcreator={LaTeX via pandoc}}
\urlstyle{same} % disable monospaced font for URLs
\usepackage[margin=1in]{geometry}
\usepackage{color}
\usepackage{fancyvrb}
\newcommand{\VerbBar}{|}
\newcommand{\VERB}{\Verb[commandchars=\\\{\}]}
\DefineVerbatimEnvironment{Highlighting}{Verbatim}{commandchars=\\\{\}}
% Add ',fontsize=\small' for more characters per line
\usepackage{framed}
\definecolor{shadecolor}{RGB}{248,248,248}
\newenvironment{Shaded}{\begin{snugshade}}{\end{snugshade}}
\newcommand{\AlertTok}[1]{\textcolor[rgb]{0.94,0.16,0.16}{#1}}
\newcommand{\AnnotationTok}[1]{\textcolor[rgb]{0.56,0.35,0.01}{\textbf{\textit{#1}}}}
\newcommand{\AttributeTok}[1]{\textcolor[rgb]{0.77,0.63,0.00}{#1}}
\newcommand{\BaseNTok}[1]{\textcolor[rgb]{0.00,0.00,0.81}{#1}}
\newcommand{\BuiltInTok}[1]{#1}
\newcommand{\CharTok}[1]{\textcolor[rgb]{0.31,0.60,0.02}{#1}}
\newcommand{\CommentTok}[1]{\textcolor[rgb]{0.56,0.35,0.01}{\textit{#1}}}
\newcommand{\CommentVarTok}[1]{\textcolor[rgb]{0.56,0.35,0.01}{\textbf{\textit{#1}}}}
\newcommand{\ConstantTok}[1]{\textcolor[rgb]{0.00,0.00,0.00}{#1}}
\newcommand{\ControlFlowTok}[1]{\textcolor[rgb]{0.13,0.29,0.53}{\textbf{#1}}}
\newcommand{\DataTypeTok}[1]{\textcolor[rgb]{0.13,0.29,0.53}{#1}}
\newcommand{\DecValTok}[1]{\textcolor[rgb]{0.00,0.00,0.81}{#1}}
\newcommand{\DocumentationTok}[1]{\textcolor[rgb]{0.56,0.35,0.01}{\textbf{\textit{#1}}}}
\newcommand{\ErrorTok}[1]{\textcolor[rgb]{0.64,0.00,0.00}{\textbf{#1}}}
\newcommand{\ExtensionTok}[1]{#1}
\newcommand{\FloatTok}[1]{\textcolor[rgb]{0.00,0.00,0.81}{#1}}
\newcommand{\FunctionTok}[1]{\textcolor[rgb]{0.00,0.00,0.00}{#1}}
\newcommand{\ImportTok}[1]{#1}
\newcommand{\InformationTok}[1]{\textcolor[rgb]{0.56,0.35,0.01}{\textbf{\textit{#1}}}}
\newcommand{\KeywordTok}[1]{\textcolor[rgb]{0.13,0.29,0.53}{\textbf{#1}}}
\newcommand{\NormalTok}[1]{#1}
\newcommand{\OperatorTok}[1]{\textcolor[rgb]{0.81,0.36,0.00}{\textbf{#1}}}
\newcommand{\OtherTok}[1]{\textcolor[rgb]{0.56,0.35,0.01}{#1}}
\newcommand{\PreprocessorTok}[1]{\textcolor[rgb]{0.56,0.35,0.01}{\textit{#1}}}
\newcommand{\RegionMarkerTok}[1]{#1}
\newcommand{\SpecialCharTok}[1]{\textcolor[rgb]{0.00,0.00,0.00}{#1}}
\newcommand{\SpecialStringTok}[1]{\textcolor[rgb]{0.31,0.60,0.02}{#1}}
\newcommand{\StringTok}[1]{\textcolor[rgb]{0.31,0.60,0.02}{#1}}
\newcommand{\VariableTok}[1]{\textcolor[rgb]{0.00,0.00,0.00}{#1}}
\newcommand{\VerbatimStringTok}[1]{\textcolor[rgb]{0.31,0.60,0.02}{#1}}
\newcommand{\WarningTok}[1]{\textcolor[rgb]{0.56,0.35,0.01}{\textbf{\textit{#1}}}}
\usepackage{graphicx}
\makeatletter
\def\maxwidth{\ifdim\Gin@nat@width>\linewidth\linewidth\else\Gin@nat@width\fi}
\def\maxheight{\ifdim\Gin@nat@height>\textheight\textheight\else\Gin@nat@height\fi}
\makeatother
% Scale images if necessary, so that they will not overflow the page
% margins by default, and it is still possible to overwrite the defaults
% using explicit options in \includegraphics[width, height, ...]{}
\setkeys{Gin}{width=\maxwidth,height=\maxheight,keepaspectratio}
% Set default figure placement to htbp
\makeatletter
\def\fps@figure{htbp}
\makeatother
\setlength{\emergencystretch}{3em} % prevent overfull lines
\providecommand{\tightlist}{%
  \setlength{\itemsep}{0pt}\setlength{\parskip}{0pt}}
\setcounter{secnumdepth}{-\maxdimen} % remove section numbering

\author{}
\date{\vspace{-2.5em}}

\begin{document}

\hypertarget{programming-assignment-2-lexical-scoping}{%
\section{Programming Assignment 2: Lexical
Scoping}\label{programming-assignment-2-lexical-scoping}}

\begin{Shaded}
\begin{Highlighting}[]
\NormalTok{knitr}\OperatorTok{::}\NormalTok{opts\_chunk}\OperatorTok{$}\KeywordTok{set}\NormalTok{(}\DataTypeTok{echo =} \OtherTok{TRUE}\NormalTok{, }\DataTypeTok{results =} \StringTok{"asis"}\NormalTok{)}
\end{Highlighting}
\end{Shaded}

Intructions\\
Second programming assignment will require you to write an R function is
able to cache potentially time-consuming computations. For example,
taking the mean of a numeric vector is typically a fast operation.
However, for a very long vector, it may take too long to compute the
mean, especially if it has to be computed repeatedly (e.g.~in a loop).
If the contents of a vector are not changing, it may make sense to cache
the value of the mean so that when we need it again, it can be looked up
in the cache rather than recomputed. In this Programming Assignment will
take advantage of the scoping rules of the R language and how they can
be manipulated to preserve state inside of an R object. Review criteria

This assignment will be graded via peer assessment. During the
evaluation phase, you must evaluate and grade the submissions of at
least 4 of your classmates. If you do not complete at least 4
evaluations, your own assignment grade will be reduced by 20\%. Example:
Caching the Mean of a Vector In this example we introduce the
\textless\textless- operator which can be used to assign a value to an
object in an environment that is different from the current environment.
Below are two functions that are used to create a special object that
stores a numeric vector and cache's its mean.

The first function, makeVector creates a special ``vector'', which is
really a list containing a function to:

\begin{enumerate}
\def\labelenumi{\arabic{enumi}.}
\tightlist
\item
  set the value of the vector
\item
  get the value of the vector
\item
  set the value of the mean
\item
  get the value of the mean
\end{enumerate}

\begin{Shaded}
\begin{Highlighting}[]
\NormalTok{makeVector \textless{}{-}}\StringTok{ }\ControlFlowTok{function}\NormalTok{(}\DataTypeTok{x =} \KeywordTok{numeric}\NormalTok{()) \{}
\NormalTok{        m \textless{}{-}}\StringTok{ }\OtherTok{NULL}
\NormalTok{        set \textless{}{-}}\StringTok{ }\ControlFlowTok{function}\NormalTok{(y) \{}
\NormalTok{                x \textless{}\textless{}{-}}\StringTok{ }\NormalTok{y}
\NormalTok{                m \textless{}\textless{}{-}}\StringTok{ }\OtherTok{NULL}
\NormalTok{        \}}
\NormalTok{        get \textless{}{-}}\StringTok{ }\ControlFlowTok{function}\NormalTok{() x}
\NormalTok{        setmean \textless{}{-}}\StringTok{ }\ControlFlowTok{function}\NormalTok{(mean) m \textless{}\textless{}{-}}\StringTok{ }\NormalTok{mean}
\NormalTok{        getmean \textless{}{-}}\StringTok{ }\ControlFlowTok{function}\NormalTok{() m}
        \KeywordTok{list}\NormalTok{(}\DataTypeTok{set =}\NormalTok{ set, }\DataTypeTok{get =}\NormalTok{ get,}
             \DataTypeTok{setmean =}\NormalTok{ setmean,}
             \DataTypeTok{getmean =}\NormalTok{ getmean)}
\NormalTok{\}}
\end{Highlighting}
\end{Shaded}

The following function calculates the mean of the special ``vector''
created with the above function. However, it first checks to see if the
mean has already been calculated. If so, it gets the mean from the cache
and skips the computation. Otherwise, it calculates the mean of the data
and sets the value of the mean in the cache via the setmean function.

\begin{Shaded}
\begin{Highlighting}[]
\NormalTok{cachemean \textless{}{-}}\StringTok{ }\ControlFlowTok{function}\NormalTok{(x, ...) \{}
\NormalTok{        m \textless{}{-}}\StringTok{ }\NormalTok{x}\OperatorTok{$}\KeywordTok{getmean}\NormalTok{()}
        \ControlFlowTok{if}\NormalTok{(}\OperatorTok{!}\KeywordTok{is.null}\NormalTok{(m)) \{}
                \KeywordTok{message}\NormalTok{(}\StringTok{"getting cached data"}\NormalTok{)}
                \KeywordTok{return}\NormalTok{(m)}
\NormalTok{        \}}
\NormalTok{        data \textless{}{-}}\StringTok{ }\NormalTok{x}\OperatorTok{$}\KeywordTok{get}\NormalTok{()}
\NormalTok{        m \textless{}{-}}\StringTok{ }\KeywordTok{mean}\NormalTok{(data, ...)}
\NormalTok{        x}\OperatorTok{$}\KeywordTok{setmean}\NormalTok{(m)}
\NormalTok{        m}
\NormalTok{\}}
\end{Highlighting}
\end{Shaded}

\hypertarget{assignment-caching-the-inverse-of-a-matrix}{%
\subsection{Assignment: Caching the Inverse of a
Matrix}\label{assignment-caching-the-inverse-of-a-matrix}}

Matrix inversion is usually a costly computation and there may be some
benefit to caching the inverse of a matrix rather than compute it
repeatedly (there are also alternatives to matrix inversion that we will
not discuss here). Your assignment is to write

Write the following functions:

\begin{enumerate}
\def\labelenumi{\arabic{enumi}.}
\tightlist
\item
  makeCacheMatrix: This function creates a special ``matrix'' object
  that can cache its inverse.
\end{enumerate}

\begin{Shaded}
\begin{Highlighting}[]
\CommentTok{\#\# A pair of functions that cache the inverse of a matrix.}
\CommentTok{\#\# This function creates a special "matrix" object that can cache its inverse.}

\NormalTok{makeCacheMatrix \textless{}{-}}\StringTok{ }\ControlFlowTok{function}\NormalTok{(}\DataTypeTok{x =} \KeywordTok{matrix}\NormalTok{()) \{}
\NormalTok{  inv \textless{}{-}}\StringTok{ }\OtherTok{NULL}
\NormalTok{  set \textless{}{-}}\StringTok{ }\ControlFlowTok{function}\NormalTok{(y)\{}
\NormalTok{    x \textless{}\textless{}{-}}\StringTok{ }\NormalTok{y}
\NormalTok{    inv \textless{}\textless{}{-}}\StringTok{ }\OtherTok{NULL}
\NormalTok{  \}}
\NormalTok{  get \textless{}{-}}\StringTok{ }\ControlFlowTok{function}\NormalTok{() x}
\NormalTok{  setInverse \textless{}{-}}\StringTok{ }\ControlFlowTok{function}\NormalTok{(solveMatrix) inv \textless{}\textless{}{-}}\StringTok{ }\NormalTok{solveMatrix}
\NormalTok{  getInverse \textless{}{-}}\StringTok{ }\ControlFlowTok{function}\NormalTok{() inv}
  \KeywordTok{list}\NormalTok{(}\DataTypeTok{set =}\NormalTok{ set, }\DataTypeTok{get =}\NormalTok{ get, }\DataTypeTok{setInverse =}\NormalTok{ setInverse, }\DataTypeTok{getInverse =}\NormalTok{ getInverse)}
\NormalTok{\}}
\end{Highlighting}
\end{Shaded}

\begin{enumerate}
\def\labelenumi{\arabic{enumi}.}
\setcounter{enumi}{1}
\tightlist
\item
  cacheSolve: This function computes the inverse of the special
  ``matrix'' returned by makeCacheMatrix above. If the inverse has
  already been calculated (and the matrix has not changed), then the
  cachesolve should retrieve the inverse from the cache. Computing the
  inverse of a square matrix can be done with the solve function in R.
  For example, if X is a square invertible matrix, then solve(X) returns
  its inverse.
\end{enumerate}

\begin{Shaded}
\begin{Highlighting}[]
\CommentTok{\#\# This function computes the inverse of the special "matrix" returned by makeCacheMatrix above.}
\NormalTok{cacheSolve \textless{}{-}}\StringTok{ }\ControlFlowTok{function}\NormalTok{(x, ...) \{}
  \CommentTok{\#\# Return a matrix that is the inverse of \textquotesingle{}x\textquotesingle{}}
\NormalTok{  inv \textless{}{-}}\StringTok{ }\NormalTok{x}\OperatorTok{$}\KeywordTok{getInverse}\NormalTok{()}
  \ControlFlowTok{if}\NormalTok{(}\OperatorTok{!}\KeywordTok{is.null}\NormalTok{(inv))\{}
    \KeywordTok{message}\NormalTok{(}\StringTok{"getting cached data"}\NormalTok{)}
    \KeywordTok{return}\NormalTok{(inv)}
\NormalTok{  \}}
\NormalTok{  data \textless{}{-}}\StringTok{ }\NormalTok{x}\OperatorTok{$}\KeywordTok{get}\NormalTok{()}
\NormalTok{  inv \textless{}{-}}\StringTok{ }\KeywordTok{solve}\NormalTok{(data)}
\NormalTok{  x}\OperatorTok{$}\KeywordTok{setInverse}\NormalTok{(inv)}
\NormalTok{  inv      }
\NormalTok{\}}
\end{Highlighting}
\end{Shaded}

For this assignment, assume that the matrix supplied is always
invertible.

In order to complete this assignment, you must do the following:

\begin{enumerate}
\def\labelenumi{\arabic{enumi}.}
\tightlist
\item
  Fork the
  \href{https://github.com/rdpeng/ProgrammingAssignment2}{GitHub
  repository containing the stub R files} to create a copy under your
  own account.
\item
  Clone your forked GitHub repository to your computer so that you can
  edit the files locally on your own machine.
\item
  Edit the R file contained in the git repository and place your
  solution in that file (please do not rename the file).
\item
  Commit your completed R file into YOUR git repository and push your
  git branch to the GitHub repository under your account.
\item
  Submit to Coursera the URL to your GitHub repository that contains the
  completed R code for the assignment.
\end{enumerate}

In addition to submitting the URL for your GitHub repository, you will
need to submit the 40 character SHA-1 hash (as string of numbers from
0-9 and letters from a-f) that identifies the repository commit that
contains the version of the files you want to submit. You can do this in
GitHub by doing the following:

\begin{enumerate}
\def\labelenumi{\arabic{enumi}.}
\tightlist
\item
  Going to your GitHub repository web page for this assignment
\item
  Click on the ``?? commits'' link where ?? is the number of commits you
  have in the repository. For example, if you made a total of 10 commits
  to this repository, the link should say ``10 commits''.
\item
  You will see a list of commits that you have made to this repository.
  The most recent commit is at the very top. If this represents the
  version of the files you want to submit, then just click the ``copy to
  clipboard'' button on the right hand side that should appear when you
  hover over the SHA-1 hash. Paste this SHA-1 hash into the course web
  site when you submit your assignment. If you don't want to use the
  most recent commit, then go down and find the commit you want and copy
  the SHA-1 hash.
\end{enumerate}

A valid submission will look something like (this is just an example!)

\url{https://github.com/rdpeng/ProgrammingAssignment2}

\hypertarget{grading}{%
\subsection{Grading}\label{grading}}

This assignment will be graded via peer assessment. During the
evaluation phase, you must evaluate and grade the submissions of at
least 4 of your classmates. If you do not complete at least 4
evaluations, your own assignment grade will be reduced by 20\%.

\end{document}
